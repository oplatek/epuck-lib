%%%%%%%%%%%%%%%%%%%%%%%%%%%%%%%%%%%%%%%%%%%%%%%%%%%%%%%%%%%%%%%%%%%%%%%%%%%%%%%%%%%%%%%%%%%%%%%%%%%%%%%%%%
\appendix
%%%%%%%%%%%%%%%%%%%%%%%%%%%%%%%%%%%%%%%%%%%%%%%%%%%%%%%%%%%%%%%%%%%%%%%%%%%%%%%%%%%%%%%%%%%%%%%%%%%%%%%%%%
\chapter{Installation guide of {\it Elib} library and its tools} \label{app:installelib}
	The following sections guide an user on his first step in e-Puck programming via {\it Elib}.
	First of all the requirements for running a sample $C\#$ program {\it TestElib} will be presented.
	Guide line how to explore and install the {\it TestElib} follows.
	The installation of {\it Elib} and {\it TestElib} differs according your operating system and runtime.
	The possibilities are Mono\cite{mono} and .Net runtimes. Mono runs on Linux as well as on Windows. Microsoft runtime .Net
	runs only under Windows, but supports also other languages than $C\#$ in contrast to Mono.

	The installation does register neither {\it Elib} nor {\it TestElib} into a operating system. It enables
	a simple copy instalation without a need of the administrator rights.
	\section{Requirements for running an e-Puck's first program}
	\label{sec:require}
	Let us suppose, that we have an e-Puck robot with {\it BTCom} 1.1.3 charged and equipped with default programmes.
	\footnote{\small{If you need download {\it BTCom} to e-Puck, follow the guide 
	\url{http://www.e-puck.org/index.php?option=com_remository&Itemid=71&func=fileinfo&id=16}.}}
	Let us start with turned off e-Puck. Turn e-Puck on.
	A green LED shines if the e-Puck is on and the battery is not empty.
	The low state of batery is signalled by a small red LED next to the green LED. 
	If the red diode is blinking or shining take off the batery and let it recharge.


	We need a Bluetooth device, because we want to send commands to e-Puck via Bluetooth. Install it and assure,
	that it is working.
	At last, we need the runtime, which is necessary to run a compiled $C\#$ programs. 
	{\it Elib} library, {\it TestElib} and {\it Elib Tools} require .Net 2.0 or a higher version of .Net under Windows.
	To run the mentioned programs under Linux use at least version 2.0 of Mono runtime.
	A graphical application {\it Elib Joystick} and simple console application Simulator
	run only under Windows and {\it Elib Joystick} needs .Net 3.5 or higher.

	All programs are aimed at programmers and therefore we recommend the following integrated development environments (IDE) ,
	which are serious benefits of $C\#$ programming.
	We published all programs in formats of a Microsoft Visual Studio 2008 solution or a MonoDevelop solution.  
	Microsoft Visual Studio 2008 (MSVS) and MonoDevelop 2.4 IDEs can be obtained freely for educational purposes.
	If at least one of the IDE is installed, all prerequisities are fulfilled to a comfort exploration of presented examples. 
	They substitute compiler, editor and debugger and save a lot of work.
	All following examples suppose, that MonoDevelop or MSVS is installed.

	Last but not least we have to check it the e-Puck's selector is at the right position.
	E-puck is deployed with {\it BTCom} downloaded to its michrochip. 
	The {\it BTCom} is saved under the second position of selector. 
	Turn selector directly to e-Puck's microphone, shift it twice to the left 
	in order to choose {\it BTCom} from default settings.

	Prepare {\it Elib} library for an installation.
	
	\section{copy installation} \label{sec:copy}
	Let us suppose that all preconditions from section ~\ref{sec:require} are met.
	Let us remind of the content of the {\it Elib} package.\\

	Both .Net and Mono technologies use a solution and projects files to group a source codes of applications.
	We placed the project and solution files together with the relevant source files in following folders:

	\begin{enumerate}
		\item Folder { \sf elib} folder contains a solution with {\it Elib} project and {\it TestElib} project.
			It contains {\it Elib} library project itself too.
		\item Folder { \sf testelib} folder contains only the project console {\it TestElib}, 
		which illustrates a typical applications of {\it Elib} library on sample programs.
		\item Folder { \sf et} folder contains solution and console project {\it Elib Tools}. 
		It allows process log files generated by programms,
		which use {\it Elib}.
		\item Folder { \sf joystick} hides a solution of a graphical application {\it Elib Joystick},
		which introduces e-Puck without programming. On the other hand, it shows
		how to build a Wpf application, which uses {\it Elib} and its multithreaded asynchronous model.
		\item { \sf simulator} folder contains a very simple windows console application Simulator and its solution. 
		Simulator requires VSPE emulator of serial port. Simulator
		 substitutes e-Puck by repeating constant answers. It allows to test your application without e-Puck.
	\end{enumerate}

	Most significant parts are located in the first three folders.
	{\it Elib Joystick's} crucial programming technique are described in Appendix ~\ref{app:joystick}.
	The only other purpose of {\it Elib Joystick} is to provide a toy, which can access e-Puck's sensors and actuators without no knowlegde of progamming.
	The Simulator application is on the other hand a simple tool for programmers, who profile their application for e-Puck for a better performance.
	See its code for instructions how to use it and modify it.

	The applications {\sf et}, {\sf {\it Elib Joystick}}, {\sf Simulator}, {\sf Elib} and {\sf {\it TestElib}}
	can be installed to your computer only by copying.
	If you are using MonoDevelop, choose folders with suffix "$\_monodev$". 
	Copy the appropriete folder to the desired location.
	In order to see the source files, build them and run them open the downloaded folder, select the solution file with suffix "$sln$" and open it with your IDE.
	Build the solution file in MSVS by pressing Ctrl + B and in MonoDevelop by pressing F7. 
	In order to run the program use Ctrl + F5 in both IDEs.
	{\it Elib Joystick} and {\it TestElib} applications use a serial port to a communication with e-Puck. 
	The serial port has to be configured before running the applications.

	In order to configure the serial port turn your Bluetooth device and e-Puck on. 
	The devices can be paired together using an operating system specific application 
	for example Bluetooth Places on Windows and Bluetooth Manager on Linux. 
	The applications creates a virtual serial port, but before connecting to the port
	the devices has to be paired.
	The process of pairing requires a four digit number, 
	which is the single number printed on e-Puck's body.
	After pairing the devices, connect to the serial port.
	The bluetooth application assigns a serial port name to the e-Puck robot. 
	On Windows the port name looks like 'COM3'. 
	Linux port name has similar format like '/dev/rfcomm1'.
	The digits at the ends of port names differs according to the situation.
	In order to achive successfully run of {\it Elib Joystick} or {\it TestElib} check
	whether the assigned port name match the port name used by {\it Elib Joystick} or {\it TestElib}.
	If the assigned port name differs from the port name used in {\it Elib Joystick}
	or in {\it TestElib} application, then replace the port name used in the applications by
	the new port name assigned by operating system.

	We believe, that all conditions for a successful use  of {\it Elib} has been presented. 
	Let us note, that the most problems are caused by the hardware.
	Assure that you switch both e-Puck and Bluetooth on. Take in mind, 
	that the red control for indicating e-Puck's low battery is not reliable,
	so we suggest fully charging the battery before the first use.

%%%%%%%%%%%%%%%%%%%%%%%%%%%%%%%%%%%%%%%%%%%%%%%%%%%%%%%%%%%%%%%%%%%%%%%%%%%%%%%%%%%%%%%%%%%%%%%%%%%%%%%%%%
%%%%%%%%%%%%%%%%%%%%%%%%%%%%%%%%%%%%%%%%%%%%%%%%%%%%%%%%%%%%%%%%%%%%%%%%%%%%%%%%%%%%%%%%%%%%%%%%%%%%%%%%%%
%\chapter{Elib Joystick} \label{app:joystick}
	\label{app:joystick} 
	Elib Joystick is a graphical application for controlling e-Puck via Bluetooth.
	It accesses all sensors and actuators of e-Puck and allows user to control e-Puck interactively.
	Is inspired by Epuck Monitor \cite{monitor}, but it is written in $C\#$ and uses asynchronous model
	of Elib to be more responsive than Epuck Monitor.

	\section{User Guide} \label{sec:joyguide}
	Elib Joystick graphical user interface is very simple. It is divided into several areas.
	See ~\ref{pic:joystick_start} figure.
	\input{joystick_start.TpX}

	In the middle is a connection panel. The connection panel is the only panel, which 
	has active controls after start up. To activate the rest of controls it is necessary to
	pair e-Puck and connected to an assigned port. See ~\ref{sec:connect} guide. 
	\input{joystick_log.TpX}
	The only function function available in non connected state is Logging Dialogue depicted in ~\ref{pic:joystick_log}.
	In the Open Dialogue can be specified the file, where are logged all commands sent to e-Puck after the connection
	to e-Puck is established.

	In order to start using the application connect it to right serial port and press connect button.
	Automatically the position of selector is retrieved and the session is established.

	At the beginning all sensors has a uninitialised value, which is in represented as a question mark in most cases.
	For retrieving the sensors value press the relevant button and then the values are displayed in the nearest label.
	There are two exceptions. First exception is the visualization of IR sensors. Values of IR sensors
	are represented as the blue levels under checkboxes on 
	the perimeter of the virtual e-Puck.
	Second  exception is a captured picture, which is displayed enlarged on the right side after its delivery.

	If the connection is too slow or the the connection is lost, the application
	is set to initial state. The e-Puck has to be reconnected in order to continue control the robot.

	Let us remark that checking the Refresh check box can cause the application to be a little irresponsive.	
	It is so, because we wait on all values of sensors including an image synchronously.
	On contrary a usual capture and a usual transfer of an image is performed asynchronously, so 
	we presents both attitudes.

	
	\input{joystick_ko.TpX} %lost connection se zachycenym obrazkem a ruznymi hodnotami.
	In figure ~\ref{pic:joystick_ko} is shown the MessageBox, which tells the user to reconnect the application
	in order to continue.

	The actuators of e-Puck can be controlled either by clicking on the appropriate button or checkbox. Another
	possibilities are choose a value from combobox or specified the  convenient values to textboxes and press
	$set$ button.

	If you want to change the session press the $Disconnect$ button, choose another port and connect again or
	just quit the application and restart it.

	\section{Elib usage in Elib Joystick}\label{sec:joystick_trick}
	Elib Joystick is a graphical application and presents no interesting algorithms. It also uses Elib
	really simply in most cases, but there is specific problem if the callbacks are used.
	In Elib Joystick we use callbacks to retrieve an image, because it can last a long time.
	Problems in graphical application arise if a graphical interface is accessed by thread different from
	main thread, where a message loop of graphical application is running. The problem is solved,
	if all functions use the same thread in so called Single Thread Appartment(STA).

	Elib Joystick challanges this problem, because it presents a picture from e-Puck's camera, which is updated
	in $Label$ of Elib Joystick's window. The callback, which updates the image, runs in different thread.
	The problem is solved by using a $Dispatcher$ class. $Dispatcher$ acceppts functions, which updates
	the GUI from different threads and calls them on the main thread in order to maintain STA character
	of GUI. 

\begin{figure}[!hbp]
\begin{lstlisting}
//Ep is instance of e-Puck,
Ep.BeginGetImage(imgto,
  //the callback(lambda function), which is called after an image capture
  (ar) => {
    try {
      Bitmap bmp = Ep.EndGetImage(ar);
      //delegate, which enwrap the lambda function, which updates the GUI
      updateUIDelegate d = delegate {
      	//body of the callback, which updates the GUI
	pic.Source = Convert2BitmapSource(bmp,colorful);
      };
      //updating the GUI by passing the update GUI callback to guid Dispatcher
      guid.Invoke(DispatcherPriority.Normal,d);
    } catch (ElibException) {
      notConfirmedCommand(this);
    }
  }, 
  null);
\end{lstlisting}
\caption{Update of an image using $Dispatcher$}
\label{updispatcher}
\end{figure}


	The code snippets ~\ref{updispatcher} and ~\ref{upsynchronous} present usage of the $Dispatcher$ class and 
	synchronous implementation of GUI update. Synchronous update does not require the $Dispatcher$ class,
	because it is performed from main thread.

	The $guid$ $Dispatcher$ used in ~\ref{updispatcher} snippet is obtained from the main thread and
	allows the passed functions to access objects like they will be called from the main thread.

\begin{figure}[!hbp]
\begin{lstlisting}
IAsyncResult ar = Ep.BeginGetImage(imgto, null, null);
Bitmap bmp = Ep.EndGetImage(ar);
//pic is a Label
pic.Source = Convert2BitmapSource(bmp,colorful);
\end{lstlisting}
\caption{Synchronous update of an image from main thread.}
\label{upsynchronous}
\end{figure}

 %\chapter{{\it Elib Joystick}} \label{app:joystick}
\chapter{{\it Elib Joystick}} \label{app:joystick}
	{\it Elib Joystick} is a graphical application for controlling e-Puck via Bluetooth.
	It accesses all sensors and actuators of e-Puck and allows user to control e-Puck interactively.
	It is inspired by Epuck Monitor 
	\footnote{\small{Download at \url{http://www.gctronic.com/files/e-puckMonitor_2.0_code.rar}}},
	but it is written in $C\#$, and uses asynchronous model
	of {\it Elib} to be more responsive than Epuck Monitor.

	\section{User Guide} \label{sec:joyguide}
	{\it Elib Joystick} graphical user interface is very simple. The single window of {\it Elib Joystick} 
	is divided into several areas.
	See Figure ~\ref{pic:joystick_start}.
	\input{joystick_start.TpX}

	In the middle is a connection panel. The connection panel is the only panel, which 
	has actived controls after start up. To activate the rest of controls following actions are necessary to
	perform. At first pair e-Puck with your computer. Then connect the assigned virtual port with e-Puck.
	Fill the assigned port's name in text box above the "Connect" button and press
	the "Connect" button. After pressing the "Connect" button {\it Elib Joystick} is connected to e-Puck. 
	%todo proc anglicky nejde {\it Elib Joystick} connects to e-Puck?
	The rest of controls are immediately activated
	as soon as the {\it Elib Joystick} is connected to e-Puck.
	See ~\ref{sec:copy} guide for detail informations. 
	
	\input{joystick_log.TpX}
	The single function available in a non connected state
	is Open File Dialogue depicted on ~\ref{pic:joystick_log}.
	In the  Open File Dialogue can be specified the file, 
	where are logged all commands sent to e-Puck after the connection
	to e-Puck is established.

	In order to start using the application connect it to the right serial port and press the "Connect" button.
	Automatically the position of selector is retrieved and the session is established.

	At the beginning all sensors has an uninitialised value, which is represented as a question mark 
	in most cases.
	Press the relevant button for retrieving the sensors value, then the values are displayed in the nearest label.
	There are two exceptions. The first exception is the visualization of e-Puck's IR sensors. 
	The values of the IR sensors
	are represented as the blue levels under check boxes on 
	the perimeter of the virtual e-Puck.
	The second  exception is the captured picture, which is displayed enlarged on the right side after its delivery.

	If the connection is too slow or the connection is lost, the application
	is set to initial state. The e-Puck has to be reconnected in order to continue control the robot.

	Let us remark that checking the Refresh check box can cause the application to be a little unresponsive.	
	It is so, because the application waits on all values of sensors including an image synchronously.
	On the other hand, a capture and transfer of an image, which was invoked by pressing "Get Pic" button,
	is performed asynchronously. 
	We have implemented both variants in order to compare the different attitudes.

	
	\input{joystick_ko.TpX} %lost connection se zachycenym obrazkem a ruznymi hodnotami.
	In Figure ~\ref{pic:joystick_ko} is shown the MessageBox, which tells the user to reconnect the application
	in order to continue.

	The actuators of e-Puck can be controlled either by clicking on the appropriate button or check box. 
	Another	possibility is to choose a value from combo box.
	The last option is to specify the  convenient values to text boxes and press relevant $set$ button.

	If you want to change the session press the $Disconnect$ button, choose some another port and 
	press again "Connect" button or	just quit the application and restart it.

	\section{{\it Elib} usage in {\it Elib Joystick}}\label{sec:joystick_trick}
	{\it Elib Joystick} is a graphical application and does not use any sophisticated algorithms. 
	It also uses {\it Elib} really simply in most cases, but there is a crucial problem if the callbacks are used.

	In {\it Elib Joystick} we use callbacks to retrieve an image, because it can last a long time.
	Problems in a graphical application arise if a graphical interface is accessed by a thread different from
	the main thread, where a message loop of the graphical application is running. The problem is solved,
	if all functions use the same thread in the so called Single Thread Appartment(STA).

	{\it Elib Joystick} challenges this problem, because it presents a picture from e-Puck's camera, which is updated
	in $Label$ of {\it Elib Joystick's} window. The callback, which updates the image, runs in a different thread.
	The problem is solved by using a $Dispatcher$ class. $Dispatcher$ accepts functions, which updates
	the GUI from different threads and calls them on the main thread in order to maintain STA character
	of GUI. 

\begin{figure}[!hbp]
\begin{lstlisting}
//Ep is instance of e-Puck,
Ep.BeginGetImage(imgto,
  //the callback(lambda function), which is called after an image capture
  (ar) => {
    try {
      Bitmap bmp = Ep.EndGetImage(ar);
      //delegate, which enwrap the lambda function, which updates the GUI
      updateUIDelegate d = delegate {
      	//body of the callback, which updates the GUI
	pic.Source = Convert2BitmapSource(bmp,colorful);
      };
      //updating the GUI by passing the update GUI callback to guid Dispatcher
      guid.Invoke(DispatcherPriority.Normal,d);
    } catch (ElibException) {
      notConfirmedCommand(this);
    }
  }, 
  null);
\end{lstlisting}
\caption{Update of an image using $Dispatcher$}
\label{updispatcher}
\end{figure}


	The code snippets ~\ref{updispatcher} and ~\ref{upsynchronous} present usage of the $Dispatcher$ class and 
	synchronous implementation of GUI update. Synchronous update does not require the $Dispatcher$ class,
	because it is performed from main thread.

	The $guid$ $Dispatcher$ used in ~\ref{updispatcher} snippet is obtained from the main thread and
	allows the passed functions to access objects as if they are be called from the main thread.

\begin{figure}[!hbp]
\begin{lstlisting}
IAsyncResult ar = Ep.BeginGetImage(imgto, null, null);
Bitmap bmp = Ep.EndGetImage(ar);
//pic is a Label
pic.Source = Convert2BitmapSource(bmp,colorful);
\end{lstlisting}
\caption{Synchronous update of an image from the main thread.}
\label{upsynchronous}
\end{figure}

%%%%%%%%%%%%%%%%%%%%%%%%%%%%%%%%%%%%%%%%%%%%%%%%%%%%%%%%%%%%%%%%%%%%%%%%%%%%%%%%%%%%%%%%%%%%%%%%%%%%%%%%%%
%\chapter{Elib library reference documentation} \label{app:elibref}
\chapter*{Class Index}
\section{Class Hierarchy}
This inheritance list is sorted roughly, but not completely, alphabetically:\begin{DoxyCompactList}
\item \contentsline{section}{Elib.AsyncResultNoResult}{\pageref{class_elib_1_1_async_result_no_result}}{}
\begin{DoxyCompactList}
\item \contentsline{section}{Elib.AsyncResult$<$ TResult $>$}{\pageref{class_elib_1_1_async_result_3_01_t_result_01_4}}{}
\end{DoxyCompactList}
\item \contentsline{section}{Elib.ElibException}{\pageref{class_elib_1_1_elib_exception}}{}
\begin{DoxyCompactList}
\item \contentsline{section}{Elib.ArgsException}{\pageref{class_elib_1_1_args_exception}}{}
\begin{DoxyCompactList}
\item \contentsline{section}{Elib.CommandArgsException}{\pageref{class_elib_1_1_command_args_exception}}{}
\end{DoxyCompactList}
\item \contentsline{section}{Elib.SerialPortException}{\pageref{class_elib_1_1_serial_port_exception}}{}
\item \contentsline{section}{Elib.TimeoutElibException}{\pageref{class_elib_1_1_timeout_elib_exception}}{}
\item \contentsline{section}{Elib.UnconnectedException}{\pageref{class_elib_1_1_unconnected_exception}}{}
\end{DoxyCompactList}
\item \contentsline{section}{Elib.Epuck}{\pageref{class_elib_1_1_epuck}}{}
\item \contentsline{section}{Elib.Sercom}{\pageref{class_elib_1_1_sercom}}{}
\end{DoxyCompactList}

\chapter*{Class Index}
\section{Class List}
Here are the classes, structs, unions and interfaces with brief descriptions:\begin{DoxyCompactList}
\item\contentsline{section}{\hyperlink{class_elib_1_1_args_exception}{Elib.ArgsException} (\hyperlink{class_elib_1_1_args_exception}{ArgsException} is thrown if wrong arguments are passed to function in Elib )}{\pageref{class_elib_1_1_args_exception}}{}
\item\contentsline{section}{\hyperlink{class_elib_1_1_async_result_3_01_t_result_01_4}{Elib.AsyncResult$<$ TResult $>$} (Class used in System.IAsyncResult when the \char`\"{}End\char`\"{} function returns an answer. E.g M:Epuck.EndGetFtion(IAsyncResult) )}{\pageref{class_elib_1_1_async_result_3_01_t_result_01_4}}{}
\item\contentsline{section}{\hyperlink{class_elib_1_1_async_result_no_result}{Elib.AsyncResultNoResult} (Class used in System.IAsyncResult for example in \hyperlink{class_elib_1_1_epuck_a4e1dcc90a8562f5c12d0bb740e85095c}{Epuck.EndFtion} It does not allow to return an answer )}{\pageref{class_elib_1_1_async_result_no_result}}{}
\item\contentsline{section}{\hyperlink{class_elib_1_1_command_args_exception}{Elib.CommandArgsException} (Thrown if command to e-\/Puck has nonsense values )}{\pageref{class_elib_1_1_command_args_exception}}{}
\item\contentsline{section}{\hyperlink{class_elib_1_1_elib_exception}{Elib.ElibException} (If some unusual situation happends and can be recovered, then ElibEception is thrown )}{\pageref{class_elib_1_1_elib_exception}}{}
\item\contentsline{section}{\hyperlink{class_elib_1_1_epuck}{Elib.Epuck} (A virtual representation of a e-\/Puck, which allows control the robot with its function )}{\pageref{class_elib_1_1_epuck}}{}
\item\contentsline{section}{\hyperlink{class_elib_1_1_sercom}{Elib.Sercom} (\hyperlink{class_elib_1_1_sercom}{Sercom} wraps serial communication with epuck. Main qoal is to keep application responsive, although serial communication could be very irresponsive )}{\pageref{class_elib_1_1_sercom}}{}
\item\contentsline{section}{\hyperlink{class_elib_1_1_serial_port_exception}{Elib.SerialPortException} (If SerialPort throws any exception, than this exception wraps the original exception. After that the \hyperlink{class_elib_1_1_serial_port_exception}{SerialPortException} is thrown )}{\pageref{class_elib_1_1_serial_port_exception}}{}
\item\contentsline{section}{\hyperlink{class_elib_1_1_timeout_elib_exception}{Elib.TimeoutElibException} (The \hyperlink{class_elib_1_1_timeout_elib_exception}{TimeoutElibException} is thrown if the \char`\"{}End\char`\"{} function implementing IAsyncResult was called and indicates that the answer to command has not been delivered in time )}{\pageref{class_elib_1_1_timeout_elib_exception}}{}
\item\contentsline{section}{\hyperlink{class_elib_1_1_unconnected_exception}{Elib.UnconnectedException} (Thrown if session with e-\/Puck has not started or has already ended )}{\pageref{class_elib_1_1_unconnected_exception}}{}
\end{DoxyCompactList}

\chapter*{Class Documentation}
\input{class_elib_1_1_args_exception}
\include{class_elib_1_1_async_result_3_01_t_result_01_4}
\include{class_elib_1_1_async_result_no_result}
\include{class_elib_1_1_command_args_exception}
\include{class_elib_1_1_elib_exception}
\include{class_elib_1_1_epuck}
\include{class_elib_1_1_sercom}
\include{class_elib_1_1_serial_port_exception}
\include{class_elib_1_1_timeout_elib_exception}
\include{class_elib_1_1_unconnected_exception}
%\label{app:epuckref}
\chapter{$Epuck$ class reference documentation} \label{app:epuckref}
This Appendix describes the most significant class of {\it Elib}, the $Epuck$ class,
and its methods and field members.
The hierarchy of exceptions, which can be thrown by $Epuck$ class, is also included
The complete reference documentation can be found on CD, which is enclosed
to this thesis.

\chapter*{$Epuck$ class}
\chapter{$Epuck$ class reference documentation} \label{app:epuckref}
This Appendix describes the most significant class of {\it Elib}, the $Epuck$ class,
and its methods and field members.
The hierarchy of exceptions, which can be thrown by $Epuck$ class, is also included
The complete reference documentation can be found on CD, which is enclosed
to this thesis.

\chapter*{$Epuck$ class}
%from class_elib_1_1_epuck.tex generated by doxygen, which used file saved at path ../trunck/elib/docs/doxyfile
%selection from first line to  line \subsection{Detailed Description}
\hypertarget{class_elib_1_1_epuck}{
\section{Elib.Epuck Class Reference}
\label{class_elib_1_1_epuck}\index{Elib::Epuck@{Elib::Epuck}}
}


A virtual representation of e-\/Puck, which allows control the robot with its methods.  


\subsection*{Public Member Functions}
\begin{DoxyCompactItemize}
\item 
delegate void \hyperlink{class_elib_1_1_epuck_a410eea190419792987ceb49c2a7b2c9b}{OkfActuators} (object data)
\begin{DoxyCompactList}\small\item\em A format of functions, which are called if the command for an actuator is successfully confirmed in timeout. \item\end{DoxyCompactList}\item 
void \hyperlink{class_elib_1_1_epuck_a8b0942a1bbf1b78115b0eedf77e535a0}{SetEncoders} (int leftMotor, int rightMotor, OkfActuators okf, KofCallback kof, object state, double timeout)
\begin{DoxyCompactList}\small\item\em It sets the values of the encoder to its motors. \item\end{DoxyCompactList}\item 
void \hyperlink{class_elib_1_1_epuck_aa15c26b061ee31dfb343f546d89bd864}{CalibrateIRSensors} (OkfActuators okf, KofCallback kof, object state, double timeout)
\begin{DoxyCompactList}\small\item\em It calibrates the IR sensors. Usufull for proximity measurement. \item\end{DoxyCompactList}\item 
void \hyperlink{class_elib_1_1_epuck_a6c5b0e61b183ae36e78251e51e32bdd3}{Stop} (OkfActuators okf, KofCallback kof, object state, double timeout)
\begin{DoxyCompactList}\small\item\em It stops e-\/Puck. \item\end{DoxyCompactList}\item 
void \hyperlink{class_elib_1_1_epuck_a36714362701622796cf01480ea4f8d32}{Reset} (OkfActuators okf, KofCallback kof, object state, double timeout)
\begin{DoxyCompactList}\small\item\em It resets e-\/Puck. \item\end{DoxyCompactList}\item 
void \hyperlink{class_elib_1_1_epuck_a4598315e376ddca08db145d5eaa22efb}{Motors} (double leftMotor, double rightMotor, OkfActuators okf, KofCallback kof, object state, double timeout)
\begin{DoxyCompactList}\small\item\em It sets the speed of e-\/Puck's motors. \item\end{DoxyCompactList}\item 
void \hyperlink{class_elib_1_1_epuck_a2c17af7601d0b318f2272f87276c2e87}{LightX} (int num, Turn how, OkfActuators okf, KofCallback kof, object state, double timeout)
\begin{DoxyCompactList}\small\item\em It turns on or off one of 8 lights of e-\/Puck. \item\end{DoxyCompactList}\item 
void \hyperlink{class_elib_1_1_epuck_a237316be8571dde0ed2e9c0b2fca6321}{BodyLight} (Turn how, OkfActuators okf, KofCallback kof, object state, double timeout)
\begin{DoxyCompactList}\small\item\em Turn off and on the body light. \item\end{DoxyCompactList}\item 
void \hyperlink{class_elib_1_1_epuck_a17d5e4e8c2755b6eff667784aa2a12a1}{FrontLight} (Turn how, OkfActuators okf, KofCallback kof, object state, double timeout)
\begin{DoxyCompactList}\small\item\em It turns the front LED front off and on. It can produce enough light capturing an picture. \item\end{DoxyCompactList}\item 
void \hyperlink{class_elib_1_1_epuck_a03911f706081e305b8602f921ad6b34b}{SetCam} (int width, int height, Zoom zoom, CamMode mode, OkfActuators okf, KofCallback kof, object state, double timeout)
\begin{DoxyCompactList}\small\item\em It sets the parameters of a cam. Maximum size of a picture can be 3200 bytes. The picture has width$\ast$height bytes in black and white mode and it has width$\ast$height$\ast$2 bytes in colourful mode. \item\end{DoxyCompactList}\item 
void \hyperlink{class_elib_1_1_epuck_a74acd305895e4f07920c45f8b29a8157}{PlaySound} (int SoundNum, OkfActuators okf, KofCallback kof, object state, double timeout)
\begin{DoxyCompactList}\small\item\em It begins to play sound. \item\end{DoxyCompactList}\item 
void \hyperlink{class_elib_1_1_epuck_a4e1dcc90a8562f5c12d0bb740e85095c}{EndFtion} (IAsyncResult ar)
\begin{DoxyCompactList}\small\item\em It waits synchronously until the function, which created an instance of\hyperlink{class_elib_1_1_async_result_no_result}{AsyncNoResult} {\itshape ar\/} . \item\end{DoxyCompactList}\item 
int\mbox{[}$\,$\mbox{]} \hyperlink{class_elib_1_1_epuck_af67fabd4c8d5fc5a9186b37157827512}{EndGetFtion} (IAsyncResult ar)
\begin{DoxyCompactList}\small\item\em It waits synchronously until a function, which asked for sensors with  values, gets the desired answer or timeout elapses. It uses {\itshape ar\/}  ir order to get the  values. \item\end{DoxyCompactList}\item 
string \hyperlink{class_elib_1_1_epuck_a74b28dea3f43cda9a42745236a2a0910}{EndInfoFtion} (IAsyncResult ar)
\begin{DoxyCompactList}\small\item\em It waits synchronously until a function, which asked for sensors with  values, gets the desired answer or timeout elapses. It uses {\itshape ar\/}  ir order to get the  values. \item\end{DoxyCompactList}\item 
Bitmap \hyperlink{class_elib_1_1_epuck_a8b2cf6f690beeab328e5a6c533cfa2b0}{EndGetImage} (IAsyncResult ar)
\begin{DoxyCompactList}\small\item\em It waits synchronously until a function, which asked for an image, gets the System.Drawing.Bitmap or timeout elapses. It uses {\itshape ar\/}  ir order to get the System.Drawing.Bitmap. \item\end{DoxyCompactList}\item 
IAsyncResult \hyperlink{class_elib_1_1_epuck_a760541d00f318ee59be9fb1c83a0f8cd}{BeginCalibrateIRSensors} (double timeout, AsyncCallback callback, Object state)
\begin{DoxyCompactList}\small\item\em It calibrates the IR sensors. Usufull for proximity measurement. \item\end{DoxyCompactList}\item 
IAsyncResult \hyperlink{class_elib_1_1_epuck_aaa46f6da4226876036df7ba7cf2adca3}{BeginMotors} (double leftMotor, double rightMotor, double timeout, AsyncCallback callback, Object state)
\begin{DoxyCompactList}\small\item\em It sets the speed of e-\/Puck's motors. \item\end{DoxyCompactList}\item 
IAsyncResult \hyperlink{class_elib_1_1_epuck_af69a96d43cc6d5c47960061312d1937b}{BeginGetMikes} (double timeout, AsyncCallback callback, object state)
\begin{DoxyCompactList}\small\item\em It gets the current amplitude of sound.(Sound strenght\}. \item\end{DoxyCompactList}\item 
IAsyncResult \hyperlink{class_elib_1_1_epuck_afa1833fd4dfb949fec35efa8ec774cf9}{BeginGetLight} (double timeout, AsyncCallback callback, object state)
\begin{DoxyCompactList}\small\item\em It gets ambient light from IR sensors. The smaller values the greater light. \item\end{DoxyCompactList}\item 
IAsyncResult \hyperlink{class_elib_1_1_epuck_a4b66f73064942aa5736dca815a427d01}{BeginStop} (double timeout, AsyncCallback callback, Object state)
\begin{DoxyCompactList}\small\item\em It stops e-\/Puck. \item\end{DoxyCompactList}\item 
IAsyncResult \hyperlink{class_elib_1_1_epuck_a2b73b814db861ed276587db2fc124244}{BeginReset} (double timeout, AsyncCallback callback, Object state)
\begin{DoxyCompactList}\small\item\em It resets e-\/Puck. \item\end{DoxyCompactList}\item 
IAsyncResult \hyperlink{class_elib_1_1_epuck_a07dff54022189967aab3a17480813803}{BeginFrontLight} (Turn how, double timeout, AsyncCallback callback, Object state)
\begin{DoxyCompactList}\small\item\em It sets the front LED on or off. \item\end{DoxyCompactList}\item 
IAsyncResult \hyperlink{class_elib_1_1_epuck_ac5cbe768e3e85fc8c627331e539221a1}{BeginLightX} (int num, Turn how, double timeout, AsyncCallback callback, Object state)
\begin{DoxyCompactList}\small\item\em It sets the state of LED on or off. \item\end{DoxyCompactList}\item 
IAsyncResult \hyperlink{class_elib_1_1_epuck_a42ebf85e7f8c6e507ef83d5bd0818650}{BeginBodyLight} (Turn how, double timeout, AsyncCallback callback, Object state)
\begin{DoxyCompactList}\small\item\em It changes state of the body light. \item\end{DoxyCompactList}\item 
IAsyncResult \hyperlink{class_elib_1_1_epuck_a58eb06f216af8ff7a7203f0f91b010e7}{BeginGetIRData} (double timeout, AsyncCallback callback, Object state)
\begin{DoxyCompactList}\small\item\em It gets the IR data in in array of 6 ints with following meaning g IR check : 0xx, address : 0xx, data : 0xx. \item\end{DoxyCompactList}\item 
IAsyncResult \hyperlink{class_elib_1_1_epuck_a0e599fe0065ac2c3ae2ac96aede901c6}{BeginGetInfoHelp} (double timeout, AsyncCallback callback, Object state)
\begin{DoxyCompactList}\small\item\em It shows Epuck's help. \item\end{DoxyCompactList}\item 
IAsyncResult \hyperlink{class_elib_1_1_epuck_a55e442eec492fc282ae978b3ec1e78c2}{BeginSetCam} (int width, int height, Zoom zoom, CamMode mode, double timeout, AsyncCallback callback, Object state)
\begin{DoxyCompactList}\small\item\em It sets the parameters of e-\/Puck's cam. Maximum size of a picture can be 3200 bytes and is computed width$\ast$height in black and white mode and width$\ast$height$\ast$2 in colourful mode. \item\end{DoxyCompactList}\item 
IAsyncResult \hyperlink{class_elib_1_1_epuck_a2426a9ea921c131b8349204f477121f5}{BeginPlaySound} (int SoundNum, double timeout, AsyncCallback callback, Object state)
\begin{DoxyCompactList}\small\item\em It begins to play sound. \item\end{DoxyCompactList}\item 
IAsyncResult \hyperlink{class_elib_1_1_epuck_a0146f79525c327c913b540186e354916}{BeginGetInfoVersion} (double timeout, AsyncCallback callback, Object state)
\begin{DoxyCompactList}\small\item\em It gets the BTCom version. \item\end{DoxyCompactList}\item 
IAsyncResult \hyperlink{class_elib_1_1_epuck_a66741b5d59b9bb5a26596b050741dee8}{BeginGetIR} (double timeout, AsyncCallback callback, object state)
\begin{DoxyCompactList}\small\item\em It gets the proximity from IR sensors. Obstacle can be recongnized up to 4 cm. \item\end{DoxyCompactList}\item 
IAsyncResult \hyperlink{class_elib_1_1_epuck_a6d09043fbbed47c089ad6f7cffa05e8c}{BeginGetAccelerometer} (double timeout, AsyncCallback callback, object state)
\begin{DoxyCompactList}\small\item\em It returns vector of values, which indicates the slant of e-\/Puck. \item\end{DoxyCompactList}\item 
IAsyncResult \hyperlink{class_elib_1_1_epuck_a23a675feab78848b194cb49a01a30463}{BeginGetSelector} (double timeout, AsyncCallback callback, object state)
\begin{DoxyCompactList}\small\item\em It returns a selector position. \item\end{DoxyCompactList}\item 
IAsyncResult \hyperlink{class_elib_1_1_epuck_a26cd19a983186a23aa49d99106bd33d1}{BeginGetSpeed} (double timeout, AsyncCallback callback, object state)
\begin{DoxyCompactList}\small\item\em It gets the current speed of both wheels. Speed on a wheel is from -\/1 to 1. \item\end{DoxyCompactList}\item 
IAsyncResult \hyperlink{class_elib_1_1_epuck_aa9e87b03e5a8ec36db30b6340d0da861}{BeginGetCamParams} (double timeout, AsyncCallback callback, object state)
\begin{DoxyCompactList}\small\item\em It gets current camera settings. \item\end{DoxyCompactList}\item 
IAsyncResult \hyperlink{class_elib_1_1_epuck_ad4cfe3776708816e78a7c53ddde315fb}{BeginGetEncoders} (double timeout, AsyncCallback callback, object state)
\begin{DoxyCompactList}\small\item\em It gets a current state of encoders. It is measured in steps. It is nulled if the e-\/Puck resets. \item\end{DoxyCompactList}\item 
IAsyncResult \hyperlink{class_elib_1_1_epuck_afe274f8efd46c68587ec706f7dfc8a35}{BeginGetImage} (double timeout, AsyncCallback callback, object state)
\begin{DoxyCompactList}\small\item\em It gets a picture. It can take a long time. E.g. piture 40$\ast$40 in colour takes more than 0.4 s under good light conditions and with battery fully charged. \item\end{DoxyCompactList}\item 
\hyperlink{class_elib_1_1_epuck_a8c9936bf3d9add474c0ca5d2f0ea9eb4}{Epuck} (string Port, string Name)
\begin{DoxyCompactList}\small\item\em Initializes a new instance of the \hyperlink{class_elib_1_1_epuck}{Epuck} class. \item\end{DoxyCompactList}\item 
void \hyperlink{class_elib_1_1_epuck_a4f3a64ac726b05b001b49a9d4cc0dd20}{WriteToLogStream} (string comment)
\begin{DoxyCompactList}\small\item\em WriteToLogStream adds to LogStream a comment. (Before passed strings puts '\#'.). \item\end{DoxyCompactList}\item 
void \hyperlink{class_elib_1_1_epuck_a2ce884e57e8149cc35a80c0105ca0396}{Dispose} ()
\begin{DoxyCompactList}\small\item\em Performs application-\/defined tasks associated with freeing, releasing, or resetting unmanaged resources. It takes under 0.5 s. \item\end{DoxyCompactList}\item 
override string \hyperlink{class_elib_1_1_epuck_abc60a3af1ea6eb72fa127463da89e77f}{ToString} ()
\begin{DoxyCompactList}\small\item\em Returns a System.String that represents this instance. A user defined robot name and \hyperlink{class_elib_1_1_sercom}{Sercom} parameters are returned. \item\end{DoxyCompactList}\item 
void \hyperlink{class_elib_1_1_epuck_a3cf8241a923b30d480013a3f6cef3620}{StartLogging} ()
\begin{DoxyCompactList}\small\item\em 
\begin{DoxyExceptions}{Exceptions}
\item[{\em T:ElibException}]is thrown when \hyperlink{class_elib_1_1_epuck_ac62091639e5474b54082b4c92d6f9fd5}{LogStream} is {\ttfamily null}. \end{DoxyExceptions}
\item\end{DoxyCompactList}\item 
void \hyperlink{class_elib_1_1_epuck_abe3d90d57e2035e9c7766f1fdb34db82}{StopLogging} ()
\begin{DoxyCompactList}\small\item\em Disables logging. \item\end{DoxyCompactList}\item 
delegate void \hyperlink{class_elib_1_1_epuck_ae28892dce5a837c025be2d29e80ccab7}{OkfIntsSensors} (int\mbox{[}$\,$\mbox{]} ans, object data)
\begin{DoxyCompactList}\small\item\em A format of functions, which are called when a command requireing an array of {\ttfamily int} is confirmed in timeout. \item\end{DoxyCompactList}\item 
delegate void \hyperlink{class_elib_1_1_epuck_a4bfdd999c7ba6f9e1282077b2cec1cc0}{OkfStringSensors} (string ans, object data)
\begin{DoxyCompactList}\small\item\em A format of functions, which are called when a command requireing a {\ttfamily string} is confirmed in timeout. \item\end{DoxyCompactList}\item 
delegate void \hyperlink{class_elib_1_1_epuck_a0b10e3ac7ceeaee8f73bc5194366e3d4}{OkfKofCamSensor} (Bitmap ans, object data)
\begin{DoxyCompactList}\small\item\em A format of functions, which are called when a command requireing a {\ttfamily Bitmap} is confirmed in timeout. \item\end{DoxyCompactList}\item 
void \hyperlink{class_elib_1_1_epuck_a93ba5219d4d0b3ac382a902b35e7e372}{GetImage} (OkfKofCamSensor okf, OkfKofCamSensor kof, object state, double timeout)
\begin{DoxyCompactList}\small\item\em It gets a picture. It can take a long time. E.g. piture 40$\ast$40 in colour takes more than 0.4 s under good light conditions and with battery fully charged. \item\end{DoxyCompactList}\item 
void \hyperlink{class_elib_1_1_epuck_a18153b3e43432af66a9baa973b77c5bd}{GetIR} (OkfIntsSensors okf, KofCallback kof, object state, double timeout)
\begin{DoxyCompactList}\small\item\em It gets the proximity from IR sensors. Obstacle can be recongnized up to 4 cm. \item\end{DoxyCompactList}\item 
void \hyperlink{class_elib_1_1_epuck_a87e5d0c07a8d2cc6909dabbc1a79ca1d}{GetAccelerometer} (OkfIntsSensors okf, KofCallback kof, object state, double timeout)
\begin{DoxyCompactList}\small\item\em It returns vector of values, which indicates the slant of e-\/Puck. \item\end{DoxyCompactList}\item 
void \hyperlink{class_elib_1_1_epuck_a40c8bc92b67a20c3fdf891997eaf03c5}{GetSelector} (OkfIntsSensors okf, KofCallback kof, object state, double timeout)
\begin{DoxyCompactList}\small\item\em It returns a selector position. \item\end{DoxyCompactList}\item 
void \hyperlink{class_elib_1_1_epuck_a8e552122cbe0eb95ac82d63018831697}{GetSpeed} (OkfIntsSensors okf, KofCallback kof, object state, double timeout)
\begin{DoxyCompactList}\small\item\em It gets the current speed of both wheels. Speed on a wheel is from -\/1 to 1. \item\end{DoxyCompactList}\item 
void \hyperlink{class_elib_1_1_epuck_a77057f06395cb1887ca3d504f1c51d2f}{GetCamParams} (OkfIntsSensors okf, KofCallback kof, object state, double timeout)
\begin{DoxyCompactList}\small\item\em It gets current camera settings. \item\end{DoxyCompactList}\item 
void \hyperlink{class_elib_1_1_epuck_a99130d9756d311379572f65411539888}{GetLight} (OkfIntsSensors okf, KofCallback kof, object state, double timeout)
\begin{DoxyCompactList}\small\item\em It gets ambient light from IR sensors. The smaller values the greater light. \item\end{DoxyCompactList}\item 
void \hyperlink{class_elib_1_1_epuck_adad61bb9ea144ca17533e59ddf583bc8}{GetEncoders} (OkfIntsSensors okf, KofCallback kof, object state, double timeout)
\begin{DoxyCompactList}\small\item\em It gets a current state of encoders. It is measured in steps. It is nulled if the e-\/Puck resets. \item\end{DoxyCompactList}\item 
void \hyperlink{class_elib_1_1_epuck_a176a4a63393394a409220699635a9196}{GetMikes} (OkfIntsSensors okf, KofCallback kof, object state, double timeout)
\begin{DoxyCompactList}\small\item\em It gets the current amplitude of sound.(Sound strenght\}. \item\end{DoxyCompactList}\item 
void \hyperlink{class_elib_1_1_epuck_ad74c7a6d5618da33d84aee5666d990c5}{GetIRData} (OkfIntsSensors okf, KofCallback kof, object state, double timeout)
\begin{DoxyCompactList}\small\item\em It gets the IR data in in array of 3 ints converted from hex number with following meaning. g IR check : 0xx, address : 0xx, data : 0xx. \item\end{DoxyCompactList}\item 
void \hyperlink{class_elib_1_1_epuck_a1569f01c2a34bb335d2f7a12c94de891}{GetHelpInfo} (OkfStringSensors okf, KofCallback kof, object state, double timeout)
\begin{DoxyCompactList}\small\item\em It shows Epuck's help. \item\end{DoxyCompactList}\item 
void \hyperlink{class_elib_1_1_epuck_aa7ed3e58936ccd1d52596b318d5ddccc}{GetVersionInfo} (OkfStringSensors okf, KofCallback kof, object state, double timeout)
\begin{DoxyCompactList}\small\item\em It gets the BTCom version. \item\end{DoxyCompactList}\end{DoxyCompactItemize}
\subsection*{Public Attributes}
\begin{DoxyCompactItemize}
\item 
const double \hyperlink{class_elib_1_1_epuck_afef1fb969494d5992aaac3146c9b3f5f}{WheelDiameter} = 4.1
\begin{DoxyCompactList}\small\item\em units: \mbox{[}cm\mbox{]} \item\end{DoxyCompactList}\item 
const double \hyperlink{class_elib_1_1_epuck_a07d37f0deafd97a37fcea3c667ae3984}{Perch} = 5.3
\begin{DoxyCompactList}\small\item\em Distance between wheels of an e-\/Puck in cm. \item\end{DoxyCompactList}\item 
const double \hyperlink{class_elib_1_1_epuck_a1c5722494d807e6a5df5835c722d4d6e}{MaxSpeed} = 13
\begin{DoxyCompactList}\small\item\em 13cm/s is a maximum speed of e-\/Puck. In Elib 13cm/s corresponds to 1.00. From -\/1.00 to 1.00 is the speed linearly growing. \item\end{DoxyCompactList}\end{DoxyCompactItemize}
\subsection*{Static Public Attributes}
\begin{DoxyCompactItemize}
\item 
static readonly int\mbox{[}$\,$\mbox{]} \hyperlink{class_elib_1_1_epuck_ab1705fc7b8cacf430c8060ee9c0b5d93}{IRSensorsDegrees} = new int\mbox{[}8\mbox{]} \{ 10, 30, 90, 170, 190, 270, 330, 350 \}
\begin{DoxyCompactList}\small\item\em Eight Infra Red sensors are placed on the perimeter of e-\/Puck, which can be obtained on the instance e of e-\/Puck by \hyperlink{}{e.BeginGetIRSensors(..)} method or by {\ttfamily e.GetIRSensors(..)} method. IrSensorsDegrees describes the degrees measured from front(There is a cam.) As you can see most of the sensors are on the front side of e-\/Puck. \item\end{DoxyCompactList}\end{DoxyCompactItemize}
\subsection*{Protected Member Functions}
\begin{DoxyCompactItemize}
\item 
virtual void \hyperlink{class_elib_1_1_epuck_aa08e40a007e38ad47254d8f43f035798}{Dispose} (bool disposing)
\begin{DoxyCompactList}\small\item\em It releases unmanaged(Serial Port in \hyperlink{class_elib_1_1_sercom}{Sercom} and access to log file) and -\/ optionally -\/ managed resources. \item\end{DoxyCompactList}\end{DoxyCompactItemize}
\subsection*{Properties}
\begin{DoxyCompactItemize}
\item 
static string \hyperlink{class_elib_1_1_epuck_ac4765eb73ad6925a5257a9ad53ba7bda}{BTComVersion}\hspace{0.3cm}{\ttfamily  \mbox{[}get\mbox{]}}
\begin{DoxyCompactList}\small\item\em It gets the BTCom version, which is a static property. \item\end{DoxyCompactList}\item 
static string \hyperlink{class_elib_1_1_epuck_a0c2ac9894699214d11d34393f4c37681}{BTComHelp}\hspace{0.3cm}{\ttfamily  \mbox{[}get\mbox{]}}
\begin{DoxyCompactList}\small\item\em It gets the BTCom help. \item\end{DoxyCompactList}\item 
int \hyperlink{class_elib_1_1_epuck_af2dd18800c02f63e0c567abede754c1c}{Working}\hspace{0.3cm}{\ttfamily  \mbox{[}get\mbox{]}}
\begin{DoxyCompactList}\small\item\em Gets the number of unconfirmed commads called on \hyperlink{class_elib_1_1_epuck}{Epuck} instance. \item\end{DoxyCompactList}\item 
int \hyperlink{class_elib_1_1_epuck_ac6b633257eb484f1126878b343d0c850}{NotSended}\hspace{0.3cm}{\ttfamily  \mbox{[}get\mbox{]}}
\begin{DoxyCompactList}\small\item\em Gets the number of waiting commands in notSended queue to be send via Serial Communication(Bluetooth). \item\end{DoxyCompactList}\item 
string \hyperlink{class_elib_1_1_epuck_a33c661ddaae53d658ea4f36de2e0f835}{Name}\hspace{0.3cm}{\ttfamily  \mbox{[}get\mbox{]}}
\begin{DoxyCompactList}\small\item\em Gets the name specified in a constructor. \item\end{DoxyCompactList}\item 
string \hyperlink{class_elib_1_1_epuck_ac267eba98589b346ef38a5ce942eff95}{Port}\hspace{0.3cm}{\ttfamily  \mbox{[}get\mbox{]}}
\begin{DoxyCompactList}\small\item\em Gets the port specified in a constructor. \item\end{DoxyCompactList}\item 
bool \hyperlink{class_elib_1_1_epuck_a91d6eb45360b7338b65faa185f657f61}{Log}\hspace{0.3cm}{\ttfamily  \mbox{[}get\mbox{]}}
\begin{DoxyCompactList}\small\item\em Return a {\ttfamily bool} flag, which indicates whether logging is on. \item\end{DoxyCompactList}\item 
TextWriter \hyperlink{class_elib_1_1_epuck_ac62091639e5474b54082b4c92d6f9fd5}{LogStream}\hspace{0.3cm}{\ttfamily  \mbox{[}get, set\mbox{]}}
\begin{DoxyCompactList}\small\item\em Enables sets or get TextWriter of e-\/Puck, where all actions of e-\/Puck are logged if loggin is turned on. \item\end{DoxyCompactList}\end{DoxyCompactItemize}


\chapter*{Exceptions}
	{\it Elib} wraps exception, which can be thrown during its usage, in $ ElibException$ class.
	$ElibExceptions$ thrown during using of {\it Elib} can be caused by other exception.
	For example, an $System.TimeoutException$ is thrown if a program tries to connect to port, 
	which is already owned by another process.
	In {\it Elib} such situation can happens and {\it Elib} wraps this exception
	with $SerialPortException$, which is inherited from $ElibException$.
	The original $System.TimeoutException$ can be retrieved from $InnerException$ property.
	See Figure ~\ref{exceptionuse}.
\begin{figure}[!hbp]
\begin{lstlisting}
//in Elib all exception are caught like this exemplary exception
      try {
        throw new ApplicationException("My exemplary exception");
      } catch (ApplicationException e){
        throw new ElibException("Just an example", e);
      }
//........................................................................
      //retrieving original exception after catching the ElibException
      try {
        //...some functions which use Elib and 
	// which throws new ApplicationException("My examplary exception")
      } catch (ElibException e) {
        Console.WriteLine(e.InnerException.Message);
      }
\end{lstlisting}
\caption{How to retrieve the original exception?}
\label{exceptionuse}
\end{figure}
	In the example is shown how is every single exception wrapped and thrown again in {\it Elib}.
	There is also depicted a way how to extract the original exception.
	The code, which catch exceptions from {\it Elib} would print followint output:
\begin{verbatim}
	My examplary exception
\end{verbatim}

	The following snippet illustrates the structure of {\it Elib's} exceptions and
	introduces all inherited subclasses.
	The subclasses serves to differentiate the {\it Elib's} exceptions.
\begin{lstlisting}
// The ElibEception is thrown, if an unusual situation happends in Elib.
// It wraps all other exceptions, which are thrown from Elib
public class ElibException : Exception { 
	//Only constructors are implemented
}
// The TimeoutElibException is thrown if the "End" function implementing IAsyncResult was called 
// and indicates that the answer to command has not been delivered in time.
public class TimeoutElibException : ElibException { 
	//Only constructors are implemented
}
// If SerialPort throws any exception, than this exception wraps the original exception.
// After that the SerialPortException is thrown.
public class SerialPortException : ElibException {
	//Only constructors are implemented
}
// ArgsException is thrown if wrong arguments are passed to function in Elib.
public class ArgsException : ElibException {
	//Only constructors are implemented
}
// Thrown if command to e-Puck has nonsense values.
public class CommandArgsException : ArgsException {
	//Only constructors are implemented
}
// Thrown if session with e-Puck has not started or has already ended.
public class UnconnectedException : ElibException {
	//Only constructors are implemented
}

\end{lstlisting}
%%%%%%%%%%%%%%%%%%%%%%%%%%%%%%%%%%%%%%%%%%%%%%%%%%%%%%%%%%%%%%%%%%%%%%%%%%%%%%%%%%%%%%%%%%%%%%%%%%%%%%%%%%
