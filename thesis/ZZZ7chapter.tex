E-Puck is a small educational robot, which is equipped with plenty of sensors. It can be controlled either
by a program running on its dsPic processor or by a program running on PC, which use a Bluetooth wireless communication and 
a special program on e-Puck.
The present thesis is devoted to an Elib library, which controls e-Puck remotely. The contribution of the thesis is 
an Elib library together with sets of examples and tools, which makes development of programs to e-Puck easier.

We have described and implemented an Elib library, which is aimed on students of mobile robotics or other courses.
The e-Puck robot is usually not the main subject of student's interest, therefor
Elib library offers a simple interface, which is described in detail on lot of examples and it allows the students control e-Puck
without useless effort.

We have prepared an TestElib project, which is available for Microsoft Visual Studio 2008 ~\ref{msvs} and also
for MonoDevelop~\ref{monodev}. The examples of TestElib project introduce features of Elib.
The functions of the project use all of sensors and actuators of e-Puck.
TestElib also introduce simple behaviours such as "Go to light" and "Bull" behaviour. "Go to light" behaviour
mades the e-Puck robot follow the source of light. If "Bull" behaviour is running, then the e-Puck
bumps into red obstacles.
Among other examples a sample of image processig is included to TestElib project.

Furthermore with Elib is deployed Elib Joystick, which is a graphical application. It allows to play with every e-Puck's sensor and actuator.

Elib Tools is another application implemented in Elib library. It is a command line application, which can parse the log from command sended to e-Puck via and collect infromation from it.
The source code of the application is also densely commented and the Elib Tools program is  ready to be extended or modified according the needs
of programmers, who use Elib.

The Elib access sensors and actuators of e-Puck with a minimum delay. 
For most sensors and actuators the sensor values respectivly the confirmation of action
are delivered under 0.1 s.
It is sufficient for comfortable control of e-Puck robot and implementing different behaviours.

As a whole, the work fullfilled its assignments. It provides a functional library for
a remote control of e-Puck, which is introduced on well documented examples.
The Elib library can be used for future works.

\subsection*{Future works}
	Most of imperfections of Elib results from implementation of $BTCom$ program,
	which is the program deployed defaultly with e-Puck and which allows Elib to control the robot.
	Simple modification of this program will allow Elib acquire a frequency of sound from microphones.
	Also a modification of $BTCom$ would allow to use e-Puck's camera more intensively.

	Althoug there are two simulators for e-Puck available, there is a space for a third simulator for e-Puck.
	The first simulator is Webots. It is a commercial software, which requires paying annual fees.
	?Enki? is the second simulator. It s a free software, but the programming of the robot is limited to a new very simple scripting language.
	Possible solution could be incorporation Elib into some existing simulator.
