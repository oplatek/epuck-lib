\chapter{BTCom source code}\label{app:btcomsource}
	Following source code is a source code of BTCom version 1.1.3, which is defaultly downloaded in e-Puck's
	at our laboratories. Elib is designed for BTCom version 1.1.1 and is also comaptible with version 2.0
	for e-Puck with no extensions.

	Not all commands of BTCom are used in Elib, because we have concentrated on textual transfer of commands.
	A relevant source code of processing the textual commands is located between  lines 322 and  502.
	The only binary command used by Elib is command to get a picture from e-Puck's camera.
	The code begins at row 259.

	Let us note, that we added some comments to the source code. Original comments begin with two slashes,
	our comments starts with four slashes.

\begin{lstlisting}
//// version 1.1.3. Following comment is copied from version 2.0, which is a compatible version for Elib library.
/*!\mainpage BTcom programm
 * \section intro_sec Introduction
 * BTcom is a simple text-based protocol that allows the PC to control the e-puck via Bluetooth. With BtCom, the PC can read
 * the e-puck's sensors and set the e-puck's actuators. The PC can be a human using a serial communication program
 * (such as Hyperterminal) or a program using a serial port library for exemple you can use the e-puck_monitor.exe
 * locate in "tool\e-puck_monitor", a part of the transfert data can be also done in a binary mode(big endian).
 *
 * \section sect_run Compiling and running
 * First of all you have to compile the programm.
 * \n \n Flash the programm to the e-puck.
 * \warning To compile the programm, you HAVE to compile with a "large data model" option you find in built option project 
 * \n The linker command lign must be -mlarge-data 
 * \n or you set the define variable MIC_SAMP_NB in the file library\a_d\advance_ad_scan\e_ad_con.h with 100
 * \n It must look like this: #define MIC_SAMP_NB 100
 * 
 * \section sect_hyperterminal Using sercom with Hyperterminal
 * Pair the e-puck with your computer: this will create a new
 * virtual COM port that you can use to communicate with the e-puck.
 * \n With Hyperterminal, connect to this COM port and play with the e-puck serial protocol. The available commands
 * are listed in this file: http://asl.epfl.ch/epfl/education/courses/MicroInfo2/TP/sercom1.02.pdf . You can also
 * press 'H' + return to see the list of available commands.
 * 
 * \section link_sec External links
 * - http://www.e-puck.org/                 The official site of the e-puck
 * - https://gna.org/projects/e-puck/       The developpers area at gna
 * - http://lsro.epfl.ch/                   The site of the lab where the e-puck has been created
 * - http://www.e-puck.org/index.php?option=com_content&task=view&id=18&Itemid=45 The license
 *
 * \author Code: Micahael Bonani \n Doc: Jonathan Besuchet
 */
#include <p30f6014a.h>

//#define FLOOR_SENSORS	// define to enable floor sensors
#define IR_RECIEVER

#include <string.h>
#include <ctype.h>
#include <stdio.h>

#include <motor_led/e_epuck_ports.h>
#include <motor_led/e_init_port.h>
#include <motor_led/advance_one_timer/e_led.h>
#include <motor_led/advance_one_timer/e_motors.h>

#include <uart/e_uart_char.h>
#include <a_d/advance_ad_scan/e_ad_conv.h>
#include <a_d/advance_ad_scan/e_acc.h>
#include <a_d/advance_ad_scan/e_prox.h>
#include <a_d/advance_ad_scan/e_micro.h>
#include <motor_led/advance_one_timer/e_agenda.h>
#include <camera/fast_2_timer/e_po3030k.h>
#include <codec/e_sound.h>

#ifdef FLOOR_SENSORS
#include <./I2C/e_I2C_protocol.h>
#endif 

#ifdef IR_RECIEVER
#include <motor_led/advance_one_timer/e_remote_control.h>
#define SPEED_IR 600
#endif 

#define uart_send_static_text(msg) do { e_send_uart1_char(msg,sizeof(msg)-1); while(e_uart1_sending()); } while(0)
#define uart_send_text(msg) do { e_send_uart1_char(msg,strlen(msg)); while(e_uart1_sending()); } while(0)


static char buffer[52*39*2+3+80];
	
extern int e_mic_scan[3][MIC_SAMP_NB];
extern unsigned int e_last_mic_scan_id;



void calibrate_ir_sensors(int ir_delta[8])
{
	int i=0,j=0;
	int long t;
	int long tmp[8];
	e_set_led(8,1);
	for (t=0;t<2000000;++t);
	e_led_clear();
	for (t=0;t<200000;++t);


	for (;i<8;++i) {
		ir_delta[i]=0;
		tmp[i]=0;
	}

	for (;j<100;++j) {
		for (i=0;i<8;++i) {
			tmp[i]+=(e_get_prox(i));
			for (t=0;t<1000;++t);
		}
	}

	for (i=0;i<8;++i) {
		ir_delta[i]=(int)(tmp[i]/(j*1.0));
	}
	
}



int main(void) {
	char c,c1,c2,wait_cam=0;
	int	i,j,n,speedr,speedl,positionr,positionl,LED_nbr,LED_action,accx,accy,accz,selector,sound;
	int cam_mode,cam_width,cam_heigth,cam_zoom,cam_size;
	static char first=0;
	int ir_delta[8]={0,0,0,0,0,0,0,0};  // Life is not perfect, unfortunately the sensors neither... So, we have to shift them
	char *address;
	char *ptr;
#ifdef IR_RECIEVER
	char ir_move = 0,ir_address= 0, ir_last_move = 0;
#endif
	TypeAccSpheric accelero;
	e_init_port();    // configure port pins
	e_start_agendas_processing();
	e_init_motors();
	e_init_uart1();   // initialize UART to 115200 Kbaud
	e_init_ad_scan();
#ifdef FLOOR_SENSORS
	e_I2Cp_init();
#endif
#ifdef IR_RECIEVER
	e_init_remote_control();
#endif
	if(RCONbits.POR) {	// reset if power on (some problem for few robots)
		RCONbits.POR=0;
		RESET();
	}
	
	/*Cam default parameter*/
	cam_mode=RGB_565_MODE;
	cam_width=40;
	cam_heigth=40;
	cam_zoom=8;
	cam_size=cam_width*cam_heigth*2;
	e_po3030k_init_cam();
	e_po3030k_config_cam((ARRAY_WIDTH -cam_width*cam_zoom)/2,(ARRAY_HEIGHT-cam_heigth*cam_zoom)/2,cam_width*cam_zoom,cam_heigth*cam_zoom,cam_zoom,cam_zoom,cam_mode);
    e_po3030k_set_mirror(1,1);
    e_po3030k_write_cam_registers();
    
    e_acc_calibr();
    
	uart_send_static_text("\f\a"
	                      "WELCOME to the SerCom protocol on e-Puck\r\n"
	                      "the EPFL education robot type \"H\" for help\r\n");
	while(1) {
		while (e_getchar_uart1(&c)==0)
		#ifdef IR_RECIEVER
		{
			
			ir_move = e_get_data();
			ir_address = e_get_address();
			if (((ir_address ==  0)||(ir_address ==  8))&&(ir_move!=ir_last_move)){
				switch(ir_move)
				{
					case 1:
						speedr = SPEED_IR;
						speedl = SPEED_IR/2;
						break;
					case 2:
						speedr = SPEED_IR;
						speedl = SPEED_IR;
						break;
					case 3:
						speedr = SPEED_IR/2;
						speedl = SPEED_IR;
						break;
					case 4:
						speedr = SPEED_IR;
						speedl = -SPEED_IR;
						break;
					case 5:
						speedr = 0;
						speedl = 0;
						break;
					case 6:
						speedr = -SPEED_IR;
						speedl = SPEED_IR;
						break;
					case 7:
						speedr = -SPEED_IR;
						speedl = -SPEED_IR/2;
						break;
					case 8:
						speedr = -SPEED_IR;
						speedl = -SPEED_IR;
						break;
					case 9:
						speedr = -SPEED_IR/2;
						speedl = -SPEED_IR;
						break;
					case 0:
						if(first==0){
							e_init_sound();
							first=1;
						}
						e_play_sound(11028,8016);
						break;
					default:
						speedr = speedl = 0;
				}
				ir_last_move = ir_move;
				e_set_speed_left(speedl);
				e_set_speed_right(speedr);
				}
	
		}
#else 
		;
#endif
		if (c<0) { // binary mode (big endian)
			i=0;
			do {
				switch(-c) { 
				case 'A': // read acceleration sensors
					accelero=e_read_acc_spheric();
					ptr=(char *)&accelero.acceleration;
					buffer[i++]=(*ptr);
					ptr++;
					buffer[i++]=(*ptr);
					ptr++;
					buffer[i++]=(*ptr);
					ptr++;
					buffer[i++]=(*ptr);
				
					ptr=(char *)&accelero.orientation;
					buffer[i++]=(*ptr);
					ptr++;
					buffer[i++]=(*ptr);
					ptr++;
					buffer[i++]=(*ptr);
					ptr++;
					buffer[i++]=(*ptr);
		
					ptr=(char *)&accelero.inclination;
					buffer[i++]=(*ptr);
					ptr++;
					buffer[i++]=(*ptr);
					ptr++;
					buffer[i++]=(*ptr);
					ptr++;
					buffer[i++]=(*ptr);
				
					break;
				case 'D': // set motor speed
					while (e_getchar_uart1(&c1)==0);
					while (e_getchar_uart1(&c2)==0);
					speedl=(unsigned char)c1+((unsigned int)c2<<8);
					while (e_getchar_uart1(&c1)==0);
					while (e_getchar_uart1(&c2)==0);
					speedr=(unsigned char)c1+((unsigned  int)c2<<8);
					e_set_speed_left(speedl);
					e_set_speed_right(speedr);
					break;
				case 'I': // get camera image ////Elib.Comnands.c_GetImage()
					e_po3030k_launch_capture(&buffer[i+3]);
					wait_cam=1;
					buffer[i++]=(char)cam_mode&0xff;//send image parameter
					buffer[i++]=(char)cam_width&0xff;
					buffer[i++]=(char)cam_heigth&0xff;
					i+=cam_size;
					break;
				case 'L': // set LED
					while (e_getchar_uart1(&c1)==0);
					while (e_getchar_uart1(&c2)==0);
					e_set_led(c1,c2);
					break;
				case 'M': // optional floor sensors
#ifdef FLOOR_SENSORS
					e_i2cp_enable();
					for(j=0;j<6;j++) buffer[i++]= e_i2cp_read(0xC0,j);
					e_i2cp_disable();
#else
					for(j=0;j<6;j++) buffer[i++]=0;
#endif
					break;
				case 'N': // read proximity sensors
					for(j=0;j<8;j++) {
						n=e_get_prox(j);
						buffer[i++]=n&0xff;
						buffer[i++]=n>>8;
					}
					break;
				case 'O': // read light sensors
					for(j=0;j<8;j++) {
						n=e_get_ambient_light(j);
						buffer[i++]=n&0xff;
						buffer[i++]=n>>8;
					}
					break;
				case 'Q': // read encoders
                    n=e_get_steps_left();
					buffer[i++]=n&0xff;
					buffer[i++]=n>>8;
                    n=e_get_steps_right();
					buffer[i++]=n&0xff;
					buffer[i++]=n>>8;
					break;
				case 'U': // get micro buffer
					address=(char *)e_mic_scan;
					e_send_uart1_char(address,600);//send sound buffer
					n=e_last_mic_scan_id;//send last scan
					buffer[i++]=n&0xff;
					break;
				default: // silently ignored
					break;
				}
				while (e_getchar_uart1(&c)==0); // get next command
			} while(c);
			if (i!=0){
				if (wait_cam) {
					wait_cam=0;
					while(!e_po3030k_is_img_ready());
				}
				e_send_uart1_char(buffer,i); // send answer
				while(e_uart1_sending());
			}
		} else if (c>0) { // ascii mode
			while (c=='\n' || c=='\r')
				 e_getchar_uart1(&c);
			buffer[0]=c;
			i = 1;
			do if (e_getchar_uart1(&c)) 
				buffer[i++]=c;
			while (c!='\n' && c!='\r');
			buffer[i++]='\0';
			buffer[0]=toupper(buffer[0]); // we also accept lowercase letters
			switch (buffer[0]) {
			case 'A': // read accelerometer  ////Elib.Comnands.c_GetAccelerometr()
				accx=e_get_acc(0);
				accy=e_get_acc(1);
				accz=e_get_acc(2);
				sprintf(buffer,"a,%d,%d,%d\r\n",accx,accy,accz);				
				uart_send_text(buffer);
			/*	accelero=e_read_acc_spheric();
				sprintf(buffer,"a,%f,%f,%f\r\n",accelero.acceleration,accelero.orientation,accelero.inclination);				
				uart_send_text(buffer);*/
				break;
			case 'B': // set body led ////Elib.Comnands.c_BodyLed(Turn how)
				sscanf(buffer,"B,%d\r",&LED_action);
			 	e_set_body_led(LED_action);
				uart_send_static_text("b\r\n");
				break;
			case 'C': // read selector position ////Elib.Comnands.c_SelectorPos()
				selector = SELECTOR0 + 2*SELECTOR1 + 4*SELECTOR2 + 8*SELECTOR3;
				sprintf(buffer,"c,%d\r\n",selector);
				uart_send_text(buffer);
				break;
			case 'D': // set motor speed ////Elib.Comnands.c_Move(int LM, int RM)
				sscanf(buffer, "D,%d,%d\r", &speedl, &speedr);
				e_set_speed_left(speedl);
				e_set_speed_right(speedr);
				uart_send_static_text("d\r\n");
				break;
			case 'E': // read motor speed ////Elib.Comnands.c_GetSpeed()
				sprintf(buffer,"e,%d,%d\r\n",speedl,speedr);
				uart_send_text(buffer);
				break; 
			case 'F': // set front led ////Elib.Comnands.c_LedFront(Turn how)
				sscanf(buffer,"F,%d\r",&LED_action);
				e_set_front_led(LED_action);
				uart_send_static_text("f\r\n");
				break;
#ifdef IR_RECIEVER				
			case 'G': ////Elib.Comnands.c_IrData()
                  sprintf(buffer,"g IR check : 0x%x, address : 0x%x, data : 0x%x\r\n", e_get_check(), e_get_address(), e_get_data());
                  uart_send_text(buffer);
                  break;
#endif
			case 'H': // ask for help ////Elib.Comnands.c_Help()
				uart_send_static_text("\n");
				uart_send_static_text("\"A\"         Accelerometer\r\n");
				uart_send_static_text("\"B,#\"       Body led 0=off 1=on 2=inverse\r\n");
				uart_send_static_text("\"C\"         Selector position\r\n");
				uart_send_static_text("\"D,#,#\"     Set motor speed left,right\r\n");
				uart_send_static_text("\"E\"         Get motor speed left,right\r\n");
				uart_send_static_text("\"F,#\"       Front led 0=off 1=on 2=inverse\r\n");
#ifdef IR_RECIEVER
				uart_send_static_text("\"G\"         IR receiver\r\n");
#endif
				uart_send_static_text("\"H\"	     Help\r\n");
				uart_send_static_text("\"I\"         Get camera parameter\r\n");
				uart_send_static_text("\"J,#,#,#,#\" Set camera parameter mode,width,heigth,zoom(1,4 or 8)\r\n");
				uart_send_static_text("\"K\"         Calibrate proximity sensors\r\n");
				uart_send_static_text("\"L,#,#\"     Led number,0=off 1=on 2=inverse\r\n");
#ifdef FLOOR_SENSORS
				uart_send_static_text("\"M\"         Floor sensors\r\n");
#endif
				uart_send_static_text("\"N\"         Proximity\r\n");
				uart_send_static_text("\"O\"         Light sensors\r\n");
				uart_send_static_text("\"P,#,#\"     Set motor position left,right\r\n");
				uart_send_static_text("\"Q\"         Get motor position left,right\r\n");
				uart_send_static_text("\"R\"         Reset e-puck\r\n");
				uart_send_static_text("\"S\"         Stop e-puck and turn off leds\r\n");
				uart_send_static_text("\"T,#\"       Play sound 1-5 else stop sound\r\n");
				uart_send_static_text("\"U\"         Get microphone amplitude\r\n");
				uart_send_static_text("\"V\"         Version of SerCom\r\n");
				break;
			case 'I':   ////Elib.Comnands.c_GetCamPar()
				sprintf(buffer,"i,%d,%d,%d,%d,%d\r\n",cam_mode,cam_width,cam_heigth,cam_zoom,cam_size);
				uart_send_text(buffer);
				break;
			case 'J'://set camera parameter see also cam library ////Elib.Comnands.c_SetCamPar(int width, int height, CamMode mode, Zoom zoom)
				sscanf(buffer,"J,%d,%d,%d,%d\r",&cam_mode,&cam_width,&cam_heigth,&cam_zoom);
				if(cam_mode==GREY_SCALE_MODE)
					cam_size=cam_width*cam_heigth;
				else
				cam_size=cam_width*cam_heigth*2;
				e_po3030k_init_cam();
				e_po3030k_config_cam((ARRAY_WIDTH -cam_width*cam_zoom)/2,(ARRAY_HEIGHT-cam_heigth*cam_zoom)/2,cam_width*cam_zoom,cam_heigth*cam_zoom,cam_zoom,cam_zoom,cam_mode);
    			e_po3030k_set_mirror(1,1);
   				e_po3030k_write_cam_registers();
   				uart_send_static_text("j\r\n");
   				break;
			case 'K':  // calibrate proximity sensors ////Elib.Comnands.c_CalibrateIR()
				uart_send_static_text("k, Starting calibration - Remove any object in sensors range\r\n");
				calibrate_ir_sensors(ir_delta);
				uart_send_static_text("k, Calibration finished\r\n");
				break;
			case 'L': // set led
				sscanf(buffer,"L,%d,%d\r",&LED_nbr,&LED_action);
				e_set_led(LED_nbr,LED_action);
				uart_send_static_text("l\r\n");
				break;
			case 'M': // read floor sensors (optional) ////Elib does not implement it, because school robots do not have this extension.
#ifdef FLOOR_SENSORS
				e_I2Cp_enable();
				for (i=0; i<6; i++)	buffer[i] = e_I2Cp_read(0xC0,i);
				e_I2Cp_disable();
				sprintf(buffer,"m,%d,%d,%d\r\n",
				(unsigned int)buffer[1] | ((unsigned int)buffer[0] << 8),
				(unsigned int)buffer[3] | ((unsigned int)buffer[2] << 8),
				(unsigned int)buffer[5] | ((unsigned int)buffer[4] << 8));
				uart_send_text(buffer);
#else
				uart_send_static_text("m,0,0,0\r\n");
#endif
				break;
			case 'N': // read proximity sensors ////Elib.Comnands.c_Proximity()
				sprintf(buffer,"n,%d,%d,%d,%d,%d,%d,%d,%d\r\n",
				        e_get_prox(0)-ir_delta[0],e_get_prox(1)-ir_delta[1],e_get_prox(2)-ir_delta[2],e_get_prox(3)-ir_delta[3],
				        e_get_prox(4)-ir_delta[4],e_get_prox(5)-ir_delta[5],e_get_prox(6)-ir_delta[6],e_get_prox(7)-ir_delta[7]);
				uart_send_text(buffer);
				break;
			case 'O': // read ambient light sensors ////Elib.Comnands.c_Light()
				sprintf(buffer,"o,%d,%d,%d,%d,%d,%d,%d,%d\r\n",
				        e_get_ambient_light(0),e_get_ambient_light(1),e_get_ambient_light(2),e_get_ambient_light(3),
				        e_get_ambient_light(4),e_get_ambient_light(5),e_get_ambient_light(6),e_get_ambient_light(7));
				uart_send_text(buffer);
				break;
			case 'P': // set motor position ////Elib.Comnands.c_SetMotorPosition(int LM, int RM)
				sscanf(buffer,"P,%d,%d\r",&positionl,&positionr);
				e_set_steps_left(positionl);
				e_set_steps_right(positionr);
				uart_send_static_text("p\r\n");
				break;
			case 'Q': // read motor position ////Elib.Comnands.c_GetMotorPosition()
				sprintf(buffer,"q,%d,%d\r\n",e_get_steps_left(),e_get_steps_right());
				uart_send_text(buffer);
				break;
			case 'R': // reset ////Elib.Comnands.c_Reset()
				uart_send_static_text("r\r\n");
				RESET();
				break;
			case 'S': // stop ////Elib.Comnands.c_Stop()
				e_set_speed_left(0);
				e_set_speed_right(0);
				e_set_led(8,0);
				
				uart_send_static_text("s\r\n");
				break;
			case 'T': // sound ////Elib.Comnands.c_Play(int OTo6)
				sscanf(buffer,"T,%d",&sound);
				if(first==0){
					e_init_sound();
					first=1;
				}
				switch(sound)
				{
					case 1: e_play_sound(0,2112);break;
					case 2: e_play_sound(2116,1760);break;
					case 3: e_play_sound(3878,3412);break;
					case 4: e_play_sound(7294,3732);break;
					case 5: e_play_sound(11028,8016);break;
					default:
						e_close_sound();
						first=0;
						break;
				}		
				uart_send_static_text("t\r\n");
				break;
			case 'U': ////Elib.Comnands.c_Microphones()
				sprintf(buffer,"u,%d,%d,%d\r\n",e_get_micro_volume(0),e_get_micro_volume(1),e_get_micro_volume(2));
				uart_send_text(buffer);
				break;
			case 'V': // get version information ////Elib.Comnands.c_Version()
				uart_send_static_text("v,Version 1.1.3 September 2006\r\n");
				break;
			default:
				uart_send_static_text("z,Command not found\r\n");
				break;
			}
		}
	}
}
\end{lstlisting}
