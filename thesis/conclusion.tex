\chapter{Conclusion} \label{chap:conclusion}
  A remote control of e-Puck over Bluetooth, implemented in {\it Elib} library, 
  has been tested on enclosed model program {\it TestElib}. 
  Tests have shown, that communication used by {\it Elib} is fast enough
  to control e-Puck robot.
  {\it Elib} library offers easy to use interface for asynchronous programming,
  which allows programmers create more sophisticated application like {\it Epuck Joystick}.
  The goals of the thesis were accomplished.
  
  The library is well documented and several model examples introduce in detail
  a usage of {\it Elib library}.
  {\it Epuck Joystick} clearly introduces all sensors and actuators of e-Puck. 
  {\it Elib Joystick} is a~graphical application.
  It gives the user a possibility to explore e-Puck's capabilities without programming.
  In addition, source code of {\it Epuck Joystick} gives programmers a guide 
  how to implement graphical application for a program 
  that uses asynchronous programming in {\it Elib}. 

  Enclosed {\it Elib Tools} application is designed for analysing
  programs that use {\it Elib}.
  It is a command line application, which can parse the log from commands sent 
  to~e-Puck and collect information from the log.
  The source code of the application is also commented and 
  the~{\it Elib Tools} program is ready to be extended 
  or modified according needs of programmers that use {\it Elib}.
  
  All programs are written in $C\#$.  $C\#$ programming language
  allows {\it Elib} to implement help for intellisense,
  which is very popular among programmers using Microsoft Visual Studio IDE,
  as well as Monodevelop IDE. {\it Elib} library can run on Linux due to
  Mono runtime, as well on Windows due to .Net runtime.
  Last but not least, $C\#$ language is easy to learn 
  and many of students already program in $C\#$.

  As a whole, the {\it Elib library} tries to offer programmer
  of e-Puck as much comfort as possible.

\subsection*{Future works}
  Most of imperfections of {\it Elib} results from the implementation 
  of the {\it BTCom} program, which is the program deployed by default with e-Puck 
  and which allows {\it Elib} to control the robot.
  A simple modification of this program would allow {\it Elib} acquire the frequency 
  of the sound from microphones.
  Also a modification of {\it BTCom} would allow to use e-Puck's camera more intensively.

  Although there are simulators for e-Puck available, 
  there is a space for another simulator for e-Puck. To our knowledge only
  Commercial simulator Webots allows control real robot remotely from PC. 

  Hence, possible solution could be incorporation of {\it Elib} into an existing simulator.
