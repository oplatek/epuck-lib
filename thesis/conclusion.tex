\chapter{Conclusion} \label{chap:conclusion}
	%\input{7chapter.tex}
	E-Puck is a small educational robot, which is equipped with enough sensors. 
        It can be controlled either
	by a program running on its dsPic processor or by a program running on PC, 
	which use Bluetooth wireless communication and 
	a special program on e-Puck.
	The contribution of the thesis is an {\it Elib} library, which controls e-Puck remotely.
	{\it Elib} library offers sets of examples and debugging tools, 
        which facilitate development of programs for e-Puck robot.

	{\it Elib} library is aimed at students of mobile robotics or other courses.
	The e-Puck robot is usually not the main subject of student's interest, therefore
	{\it Elib} library offers a simple interface, 
        which is described in detail on lot of examples and 
	it allows the users easily control e-Puck robot.

	We have prepared {\it TestElib} console application available for Microsoft Visual Studio 2008 \cite{msvs} and also
	for MonoDevelop~\cite{monodev} IDE. The examples of {\it TestElib} project illustrate features of {\it Elib}.
	The functions of the project use all of the~sensors and actuators of e-Puck.
	{\it TestElib} contains simple behaviours such as "Go to light" and "Bull" behaviour. "Go to light" behaviour
	makes the e-Puck robot follow the source of light.
	Among others an example of image processing is included.

	Furthermore an enclosed sample graphical application {\it Elib Joystick}  
        presents e-Puck's sensors and actuators
	to users, who do not want to explore e-Puck in detail, 
        but who want to play with the robot.

	{\it Elib Tools} is an application implemented in {\it Elib library}. It is a command line application, 
	which can parse the log from commands sent to e-Puck and collect the information from the log.
	The source code of the application is also commented and the {\it Elib Tools} program is  ready to be extended 
	or modified according needs of programmers, who use {\it Elib}.

	{\it Elib} accesses sensors and actuators of e-Puck with a minimum delay. 
	For most sensors and actuators the sensor values, respectively the confirmations of actions,
	are delivered under 0.1 seconds. 
	It is sufficient for a suitable control of e-Puck robot and for an implementation of different behaviours.
	See Table ~\ref{times1} for detail results.

	As a whole, the goals of the thesis were accomplished. It provides a functional library for
	a remote control of e-Puck, which is introduced on extensively documented examples.
	The {\it Elib} library can be further extended.

\subsection*{Future works}
	Most of imperfections of {\it Elib} results from the implementation 
        of the {\it BTCom} program, which is the program deployed by default with e-Puck 
        and which allows {\it Elib} to control the robot.
	A simple modification of this program would allow {\it Elib} acquire the frequency 
        of the sound from microphones.
	Also a modification of {\it BTCom} would allow to use e-Puck's camera more intensively.

	Although there are two simulators for e-Puck available, 
        there is a space for a third simulator for e-Puck.
	The first simulator is Webots. It is an expensive commercial software with complete IDE. 
	Enki is a free and efficient 2D simulator, but there is no IDE available for Enki. 
        Integrated development environment is the main advantage 
        for a development of a basic application and therefore Enki is not
	convenient for programming applications to e-Puck.
	There is also Aseba project, which is an event driven system for controlling e-Puck robots, 
	but it uses a specialised scripting language.

	Hence, possible solution could be incorporation of {\it Elib} into an existing simulator.
